\section{\uppercase{Introduction}}
\label{sec:introduction}

\noindent Bio-signals vary widely between individuals, and their expression often changes within individuals over time. Typically, brain computer interfaces (BCI) serve as an excellent example of this phenomenon. Regular calibration and re-calibration are critical to achieving decent BCI accuracy \cite{dornhege_toward_2007,mcfarland_brain-computer_2011}.

Supervised learning algorithms have assisted systems in adapting to users' personal physiology after a calibration period. In BCI, this approach has yielded proof-of-concept systems ranging from brain-controlled keyboards and wheelchairs to prosthetic arms and hands \cite{blankertz_note_2007,millan_combining_2010,d._mattia_brain_2011,hill_practical_2014,campbell_neurophone:_2010}. 

However, in order to move BCI into broader consumer markets, systems must work with more mobile sensing equipment and wearable computing platforms. Mobile device architectures limit computational complexity relative to lab-based systems, and ergonomic considerations limit the number and quality of sensors on the device. 

In this study, we simulate a simple brain-computer interface using signals acquired from a low-cost, mobile electroencephalograph (EEG) device with a single electrode. Using a BCI that takes \textit{mental gestures} as input, we investigate how the processing of bio-signals and the strategy for user calibration can impact the computational performance, reliability and calibration time of a physiological signal classification system. 

First, we present a novel signal quantization technique in which we apply logarithmic binning to power spectrum data from an EEG electrode. We find that this technique can speed up the computational performance of a classification-based BCI by 450\% without significant detriment to the system's accuracy. 

Second, we combine this technique with a progressive user calibration strategy, in which candidate mental gestures are tested in an order designed to minimize calibration time. We calibrate 86.6\% of users to a threshold of \textit{BCI literacy} (75\% accuracy) \cite{vidaurre_towards_2010} with under five minutes of training data, and 100\% of users within half an hour. 

This paper is organized as follows. We introduce relevant background research in Section \ref{sec:related}. We present the power spectrum quantization method in Section \ref{sec:quantization}, and the data used for calibration in Section \ref{sec:data}. We then evaluate the quantization method (Section \ref{sec:quantization_eval}), and we present a time-efficient calibration strategy for our BCI apparatus (Section \ref{sec:calibration_eval}). We conclude with limitations and future research directions.
