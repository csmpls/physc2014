\section{\uppercase{Introduction}}
\label{sec:introduction}

\noindent The sensor data on which physiological systems generally rely do not cary universal meanings. While the movement of a mouse can be unambiguously mapped to the position of a cursor, the expression of biosignals varies widely  between people, and within individual users over time. The need for adaptive signal processing approaches from physiological computing applicaitons create a tradeoff between computational complexity, the time users must spend calibrating the device to their individual signals, and the classification accuracy of the resulting system.

Brain-computer interface (BCI) serves as a dramatic example of this phenomenon. Neural signals are highly variable between persons, and the nonstationary nature these signals necessitate regular calibration and re-calibration. \cite{dornhege_toward_2007,mcfarland_brain-computer_2011}. Recently, the combination of machine learning algorithms and non-invasive electroencephelographs (EEG) has yielded proof-of-concept systems ranging from brain-controlled keyboards and wheelchairs to prosthetic arms and hands \cite{blankertz_note_2007,millan_combining_2010,d._mattia_brain_2011,hill_practical_2014,campbell_neurophone:_2010}. However, these systems have not found wide adoption outside the lab: BCIs often require upward of an hour to calibrate to their users, with close supervision by researchers, and may require regular recalibration due to the nonstationary nature of EEG signals. \cite{vidaurre_fully_2006,vidaurre_co-adaptive_2011,blankertz_non-invasive_2007} Recent work has come close to eliminating traditional calibration, though these techniques have not yet been applied to consumer headsets. \cite{kindermans_true_2014}

For a physiological computing application to enjoy wider adoption ``in the wild,'' it must calibrate to individual users quickly and achieve decent information transfer rates (ITR). However, it must do so with ergonomic sensors and noisy signals, as data acquisition will occur while people are moving, walking, talking, and so on. As an added challenge, computational complexity may be limited by the mobile \& wearable computing architectures on which these systems will most likely be deployed. 

In this study, we use recordings from a single, dry electroencephalographic (EEG) sensor to simulate the calibration of a binary BCI, and investigate the effect of a novel signal extraction technique on the system's computational performance and accuracy. We find that our signal extraction technique increases the computational speed of a classification-based BCI 450\% without a significant detriment to accuracy. We find that this technique can be used to calibrate 86.6\% of users to BCI literacy in under five minutes of training data, and 100\% of users in under 15 minutes. 