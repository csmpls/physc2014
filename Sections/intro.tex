\section{\uppercase{Introduction}}
\label{sec:introduction}

\textcolor{red}{The intro starts a little sharp with a broad definition. Maybe, it would be good to give more context that relates to our results, and be more precise about what we mean by BCI.}

\noindent Brain-computer interface (BCI) systems establish a direct communcative link between the brain and an electronic system \cite{dornhege_toward_2007,mcfarland_brain-computer_2011}.  Recently, the combination of machine learning algorithms and non-invasive electroencephelographs (EEG) has yielded proof-of-concept systems ranging from brain-controlled keyboards and wheelchairs to prosthetic arms and hands \cite{blankertz_note_2007,millan_combining_2010,d._mattia_brain_2011,hill_practical_2014,campbell_neurophone:_2010}. 
% TODO: more cites on cool EEG applications?

There are many reasons why these BCI systems have not found wide adoption outside of lab settings. First of all, they often require large, complex scanning caps, which are impractical for disabled users and generally undesirable for ergonomic reasons \cite{ekandem_evaluating_2012,leeb_transferring_2013}. Meanwhile, the amount of data produced by these caps is large, which places high requirements on end-user's hardware. Finally, BCIs often require upward of an hour to calibrate to their users, with close supervision by researchers, and may require regular recalibration due to the nonstationary nature of EEG signals. \cite{vidaurre_fully_2006,vidaurre_co-adaptive_2011,blankertz_non-invasive_2007} Recent work has come close to eliminating traditional calibration, though not using ergonomic headsets. 

For BCI systems to enjoy wider adoption ``in the wild,'' they must calibrate to individual users quickly and achieve decent information transfer rates (ITR), but with fewer sensors than their lab-based counterparts, and with noisier signals, as data acquisition will occur while people are performing daily tasks, moving, walking, talking, and so on. As an added challenge, their computational firepower may be limited by the mobile \& wearable computing architectures on which they will most likely be deployed. 

In this study, we use recordings from a single, dry electroencephalographic (EEG) sensor to simulate the calibration of a simple BCI, and investigate the effect of a novel signal extraction technique on the system’s computational performance and accuracy. First, we find that our signal extraction technique significantly increases the computational speed of a classification-based BCI without a significant detriment to accuracy. Second, we find evidence that this technique can be used to build effective mental task classifiers with under a minute of training data.