\section{\uppercase{Introduction}}
\label{sec:introduction}

\noindent Physiological data do not cary universal meanings. While the movement of a mouse can be straightforwardly mapped to the position of a cursor, the expression of biosignals vary 2widely between people, and often change within individuals over time. Brain-computer interface (BCI) serves as a dramatic example of this phenomenon: the nonstationary nature of neural signals necessitate regular calibration and re-calibration within users as well. \cite{dornhege_toward_2007,mcfarland_brain-computer_2011}.

Supervised learning algorithms have enabled systems that adapt to users' physiological signals after a calibration period. In BCI, this approach has yielded proof-of-concept systems ranging from brain-controlled keyboards and wheelchairs to prosthetic arms and hands \cite{blankertz_note_2007,millan_combining_2010,d._mattia_brain_2011,hill_practical_2014,campbell_neurophone:_2010}. 

% For similar devices to find use outside of the lab, they  minimize both the size of our sensing devices and the amount of time users spend calibrating the interface. % minimal number of sensors + limited computational power on mobile devices % while still accurate and quicck-to-calibrate

% BCIs often require upward of an hour to calibrate to their users, with close supervision by researchers, and may require regular recalibration due to the nonstationary nature of EEG signals. \cite{vidaurre_fully_2006,vidaurre_co-adaptive_2011,blankertz_non-invasive_2007} Recent work has come close to eliminating traditional calibration, though these techniques have not yet been applied to consumer headsets. \cite{kindermans_true_2014}


Compared to these lab-based prototypes, systems suitable for mobile, real-world use will have fewer sensors (due to ergonomic constraints on device size) and will process noisier signals (as data acquisition will occur while people are moving, walking, talking, and so on). As an added challenge, computational complexity (measured by both storage and processing power) may be limited by the mobile \& wearable computing architectures on which these systems will most likely be deployed. 

In this study, we use recordings from a single, dry electroencephalographic (EEG) sensor to simulate the calibration of a binary BCI, and investigate the effect of a novel signal extraction technique on the system's computational performance and accuracy. We find that our signal extraction technique increases the computational speed of a classification-based BCI 450\% without a significant detriment to accuracy. This technique allows us to calibrate 86.6\% of users to a threshold of BCI control in under five minutes of training data, and to calibrate 100\% of users in under 15 minutes. 