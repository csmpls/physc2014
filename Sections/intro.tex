\section{\uppercase{Introduction}}
\label{sec:introduction}

\noindent Physiological data do not carry universal meanings. While the movement of a computer mouse can be mapped to the position of a cursor in a straightforward manner, the expression of bio-signals varies widely between individuals, and often changes within individuals over time. Brain-computer interface (BCI) serves as a good example of this phenomenon: regular calibration and re-calibration are essential due to the personal and non-stationary nature of neural signals \cite{dornhege_toward_2007,mcfarland_brain-computer_2011}.

Supervised learning algorithms have enabled systems to adapt to users' personal physiology after a calibration period. In BCI, this approach has yielded proof-of-concept systems ranging from brain-controlled keyboards and wheelchairs to prosthetic arms and hands \cite{blankertz_note_2007,millan_combining_2010,d._mattia_brain_2011,hill_practical_2014,campbell_neurophone:_2010}. 

% For similar devices to find use outside of the lab, they  minimize both the size of our sensing devices and the amount of time users spend calibrating the interface. % minimal number of sensors + limited computational power on mobile devices % while still accurate and quick-to-calibrate

% BCIs often require upward of an hour to calibrate to their users, with close supervision by researchers, and may require regular recalibration due to the nonstationary nature of EEG signals. \cite{vidaurre_fully_2006,vidaurre_co-adaptive_2011,blankertz_non-invasive_2007} Recent work has come close to eliminating traditional calibration, though these techniques have not yet been applied to consumer headsets. \cite{kindermans_true_2014}

Moving from laboratory settings into the real world, these systems will have fewer and less sensitive sensors due to cost, ergonomic and aesthetic considerations. They will also process noisier signals as data acquisition will occur while people are engaged in everyday activities, walking, talking, sleeping, and so on. As an additional challenge, computational complexity, measured by both storage and processing requirements, may be limited by the mobile and wearable computing architectures on which these systems will be deployed. 

% This leads us to pose the following research question: how fast and with how little computational power can we build a practical BCI?

In this paper, we study how the processing of physiological signals and the strategy for user calibration can impact the performance of a machine-learning based bio-signal classification system. We use signals acquired from a low-cost, mobile electroencephalograph (EEG) device with a single sensor. Prior to classification, how can we operationalize the tradeoff between computational complexity and classification accuracy at the signal processing step? Given a well-tuned classifier, is it possible to realize user calibration on the order of minutes rather than hours or days?

%How can we process EEG signals such that we minimize the computational expense of classification while maximizing the system's accuracy? Is a computationally efficient signal processing technique compatible with a user-calibration protocol that achieves ``BCI literacy'' across all subjects on the order of minutes rather than hours or days?

We propose a novel signal quantization technique that applies logarithmic binning to power spectrum data from a single EEG electrode. We find that this technique can increase the computational speed of a classification-based BCI by 450\% compared to uncompressed data without significant detriment to accuracy. In conjunction with a progressive user-calibration protocol, in which candidate mental gestures are tested ``on demand'' in order to minimize calibration time, we calibrate 86.6\% of users to a threshold of BCI control in under five minutes of training data, and 100\% of users in under 20 minutes. 

This paper is organized as follows. We discuss related works in Section \ref{sec:related}, and provide a summary of the dataset in Section \ref{sec:data}. We describe our signal quantization method in Section \ref{sec:quantization}, and quantify its effect on classifier speed and accuracy in Section \ref{sec:quantization_eval}. We evaluate a user calibration strategy in Section \ref{sec:calibration_eval} before concluding.
