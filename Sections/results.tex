\section{\uppercase{Results}}

\subsection{Experiment 1}

\begin{figure}[!h]
  \vspace{-0.2cm}
  \centering
   {\epsfig{file = Figures/1-2.png, width = 5.5cm}}
  \caption{Mean best-case accuracy among all subjects compared to log of the number of bins in training data.}
  \label{fig:fig1}
  \vspace{-0.1cm}
 \end{figure}

Overall, both longer recording times and higher bin size are correlated with higher classifier accuracy. \textit{and what is important to report about this regression?}

\begin{figure}[!h]
  \vspace{-0.2cm}
  \centering
   {\epsfig{file = Figures/avg.png, width = 5.5cm}}
  \caption{SVM test time compared to number of bins in test set feature vector.}
  \label{fig:fig2}
  % \vspace{-0.1cm}
\end{figure}

\begin{figure}[!h]
  \vspace{-0.2cm}
  \centering
   {\epsfig{file = Figures/best.png, width = 5.5cm}}
  \caption{SVM train time compared to number of bins in training set feature vectors.}
  \label{fig:fig2}
  \vspace{-0.1cm}
\end{figure}


The number of bins is positively correlated with training and testing time. While test time grows linearly with the log of the number of bins. \textit{and what is important to report about this regression?}

\subsection{Experiment 2}

\begin{figure}[!h]
  \vspace{-0.2cm}
  \centering
   {\epsfig{file = Figures/best.png, width = 5.5cm}}
  \caption{Mean best-case accuracy among all subjects compared to number of seconds of data in training set.}
  \label{fig:fig2}
  \vspace{-0.1cm}
\end{figure}

 The amount of data on which the classifier is trained is positively correlated with the classifier's accuracy on data from later recordings. \textit{and what is important to report about this regression?} The number of bins in training and testing data is also positively correlated with classifier accuracy.
