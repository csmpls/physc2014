\section{\uppercase{Effect of quantization on classifier speed and accuracy}}
%\section{\uppercase{The effect of quantization technique on classifier speed and accuracy}}
\label{sec:quantization_eval}

% We examine the effect of signal resolution, operationalized by the number of bins used in the quantization step, on our BCI's performance, measured by the SVM's training time and by the SVM's estimated accuracy. 

We hypothesize that both SVM training time and accuracy increase with number of bins, i.e., the higher the signal resolution, the higher the accuracy but the longer the training time.

%H1: SVM accuracy will decrease with resolution.
%H2: SVM training time will decrease with resolution.

%\subsection{Protocol}

In order to make an optimal binary BCI for each subject, we must find the two gestures that the SVM distinguishes most reliably. For each subject, we generate every pair of two mental gestures and cross-validate our SVM on the recordings for this pair. Given seven candidate gestures, we have a total of 21 possible gesture pairs. For every pair processed, we record mean classification accuracy across all rounds of cross-validation. We record the best-performing gesture pair for each subject, which yields the optimal pair for the binary BCI.

We perform this process multiple times, varying the signal resolution by varying the number of bins from 1 to 1024. As an additional performance audit, we measure the time needed to fit an SVM to the data for two randomly selected gesture pairs across all subjects. We repeat this process ten thousand times at different resolutions, collecting the minimum time observed in each series of attempts.

%\subsection{Results}

Figure \ref{fig:accuracy_vs_bins} shows the mean best-case accuracy of the classifier versus the number of bins. We can see that the accuracy level remains above 90\% even as we reduce the number of bins to 50. Although classifier accuracy is positively correlated with signal resolution (slope = 0.0013, R-squared = 0.773, p \textless 0.001), this effect appears only at resolutions lower than 50 bins. We find no significant difference in SVM accuracy at resolutions over 50 bins.

Figure \ref{fig:training_vs_bins} shows, in log-log scale, the SVM training time versus the number of bins. We see that the log of the classifier training time is positively correlated with the log of signal resolution (slope = 0.5, R-squared = 0.947, p \textless 0.001). We also observe an increase in variance in the data, possibly due to variability in memory read and write times, which exacerbates SVM training time at larger vector sizes (as more reads and writes are being performed).

Combining these two results, Figure \ref{fig:accuracy_vs_training} confirms the tradeoff between classifier accuracy and classifier training time. It also points to the existence of a threshold resolution at around 50 bins that provides a 450\% speed improvement over a non-quantized baseline of 1024 bins without a significant detriment to classifier accuracy.

%Resolution was also positively correlated with time to train classifier (slope = 0.5 R-squared = 0.947, p \textless .001). We compare accuracy and SVM training time directly in \ref{fig:fig1c}. Thus, we find support for H2.

\begin{figure}[!h]
%  \vspace{-0.2cm}
  \centering
   {\epsfig{file = Figures/acc-bins.png, width = 8cm}}
  \caption{Mean best-case accuracy among all subjects compared to signal resolution. At resolutions of 50 points (bins) and greater, we find no evidence of an increase in classification accuracy. }
  \label{fig:accuracy_vs_bins}
  \vspace{-0.1cm}
 \end{figure}

 \begin{figure}[!h]
  \vspace{-0.2cm}
  \centering
   {\epsfig{file = Figures/traintime-bins.png, width = 8cm}}
  \caption{Log of mean classifier training time compared to log of data resolution. The slope is 0.5, indicating that the time needed to train the classifier increases at approximately the square root of the signal resolution.}
  \label{fig:training_vs_bins}
  \vspace{-0.1cm}
 \end{figure}

\begin{figure}[!h]
  \vspace{-0.2cm}
  \centering
   {\epsfig{file = Figures/acc-traintime.png, width = 8cm}}
  \caption{ Best-case accuracy compared to the time needed to train the classifier. By decreasing the number of bins in the EEG data, we can decrease the time needed to train the support vector machine up to nine times without without significant detriment to classifier accuracy. }
  \label{fig:accuracy_vs_training}
%  \vspace{-0.1cm}
 \end{figure}

Overall, we find that relatively small feature vectors produced with our method (50 values) yield classifiers as accurate as full-resolution samples (1024 values), and that reducing vector size in this way can dramatically increase computational speed.
