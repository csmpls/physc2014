\section{\uppercase{Discussion}}

\noindent We find that logarithmic binning dramatically decreases the computational expense of EEG-based calibration and classification without a significant detriment to accuracy. Further, we find that this technique is compatible with a training strategy capable of reaching acceptable classification rates with only a few minutes of training time. Since we used generallylow-cost, consumer hardware, our technique could make commercial BCI more feasible. 

The conclusions to be drawn from this study are limited in a few regards. First, calibration and classification were performed offline, so factors involving the user interface (such as feedback) are not taken into account. We cannot be sure, for instance, that our findings with short splices of ten-recordings data will persist when a system solicits recordings of only a second or under. Furthermore, a few of our tasks (e.g. the color task) relied on exogenous stimuli, which may be impractical in naturalistic settings for ergonomic reasons. Finally, we did not compare our findings to traditional signal processing methods in EEG.

Logarithmic binning could enable co-adaptive, online BCI with as few as one dry EEG sensor, making online calibration much more performant on mobile or embedded processors with limited computational resources. Alternatively, since logarithmic binning dramatically decreases the size of data fed to the classification algorithm, the technique could allow calibration to occur “in the cloud” - the BCI could pre-process the data on board, bin it, and ship this data to a more powerful server, which could process it online. By some combination of cloud-based and on-board processing, BCIs could gain from the accuracy of computationally expensive analytics without having to perform these computations on-board.

Future work could implement a system that calibrates a BCI online. Due to the small size of binned EEG signals, such a system could use a client/server architecture in which expensive processing (such as training multiple SVMs) is offloaded from the user’s system to a more powerful processor in teh cloud. This system could be used for the calibration of direct-control BCIs by attempting to find groups of tasks for which the classifier has high discriminatory power. A similar system could be used for long-term, affective recordings as well.

By collecting EEG data in the wild, we hope to discover more about how the mind behaves outside of laboratory environments. A particular interest is the non-stationary nature of neural recordings. By analyzing chronic EEG recordings at scale, we hope to make observations about how EEG signals change their expression over time, which could enable us to build more accurate BCIs that require less frequent re-calibration.