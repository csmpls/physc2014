\section{\uppercase{Discussion}}

\noindent In this study, we investigated the effect of a signal quantization technique on the performance of a binary BCI that used a low-cost, single-channel EEG headset as input. We found that our technique allowed for a computationally efficient BCI that acheives decent simulated accuracy for all users in our dataset and boasts quick user calibration times overall. 

Specifically, we find that our quantization method decreased the computational expense of EEG-based calibration (18 ms to 2ms for SVM traintime) without a significant detriment to accuracy and, using quantized data, our opportunistic user calibration strategy acheived an average of 88.3\% accuracy across all subjects. All subjects required under fifteen minutes of calibration time, and 86.6\% of these subjects required five minutes or fewer.



\section{\uppercase{Conclusions \& Future work}}

The conclusions to be drawn from this study are limited in a few regards. First, calibration and classification were performed offline, so factors involving the user interface (such as feedback) are not taken into account. We cannot be sure, for instance, that our findings with short splices of ten-recordings data will persist when a system solicits recordings of only a second or under. Furthermore, a few of our tasks (e.g. the color task) relied on exogenous stimuli, which may be impractical in naturalistic settings for ergonomic reasons. 

Our study implies that practical BCI can be achieved with as few as one, inexpensive EEG sensor, minimal processing power and a only a few minutes of user calibration. Future work could build usable, online BCI systems to test this claim more rigorously, especially in mobile and out-of-lab environments.

Since many types of physiological data can be represented as power spectra (electrocardiography, electromyography), future work could test our quantization technique in other classification-based physiological applications (heart sensing, gesture recognition, or systems with heterogeneous sensors).

Since logarithmic binning dramatically decreases the size of physiological data in memory, this technique could allow developers to more easily ship biosignals to remote servers. Future work could explore the design space associated with the storage and transmission of physiological signals. BCI calibration, for example, could occur “in the cloud” - the client would quanitize power spectra data and quantize them using our method, then ship these compressed data to a more powerful server. 

Alternatively, the small size of quantized featuare vectors could enable long-term, pervasive recording of mental states. The data would be small enough to ship and store on a centralized server, or to store locally in a decentralized fashion. Monitoring mental (or other biophysical) activity continuously in everyday settings could yield observations about human activity \& physiology that would be difficult to observe in controlled, laboratory environments.