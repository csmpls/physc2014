\section{\uppercase{Conclusion}}
\label{sec:conclusion}

\noindent In this study, we investigate the effect of a signal quantization technique on the performance of a binary BCI that uses a low-cost, single-channel EEG headset as input. We find that our technique allows for a computationally efficient BCI that achieves good simulated accuracy for all subjects in our dataset and boasts quick user calibration times overall. 

Specifically, we find that our quantization method decreases the computational expense of EEG-based calibration (from 18 ms to 2 ms for SVM training time) without a significant detriment to accuracy and, using quantized data, our opportunistic user calibration strategy achieves an average of 88.3\% accuracy across all subjects. All subjects require under fifteen minutes of calibration time, and 86.6\% of these subjects require five minutes or fewer.

The conclusions to be drawn from this study are limited in a few regards. First, calibration and classification are performed offline, so factors involving the user interface (such as feedback) are not taken into account. We cannot be sure, for instance, that our findings based on the splicing of 10-second-long recorded data will persist when a system solicits recordings of only a second or under. Furthermore, a few of the gestures (e.g., the color gesture) relied on exogenous stimuli, which may be impractical in naturalistic settings for ergonomic reasons. 

Our study implies that practical BCI can be achieved with as few as one inexpensive EEG sensor, minimal processing power, and a only a few minutes of user calibration. Future work could build usable, online BCI systems to test this claim more rigorously, especially in mobile and out-of-lab environments.

Since many types of physiological data can be represented as power spectra (e.g., electrocardiography, electromyography), future work could test our quantization technique in other classification-based physiological applications, such as heart sensing, gesture recognition, or systems with heterogeneous sensors.

Logarithmic binning dramatically decreases the size of physiological data in memory. This technique could allow developers to more easily ship bio-signals to remote servers. Future work could explore the design space associated with the storage and transmission of physiological signals. BCI calibration, for example, could occur ``in the cloud''� where the client would quantize power spectra data using our method before shipping these compressed data to a more powerful server in the network.

Alternatively, the small size of quantized feature vectors could enable long-term, pervasive recording of mental states. The data would be small enough to ship and store on a centralized server, or to store locally in a decentralized fashion. Monitoring mental and other biophysical activity continuously in everyday settings could yield novel observations about human activity \& physiology.

