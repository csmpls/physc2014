\section{\uppercase{Conclusion}}
\label{sec:conclusion}

\noindent In this study, we investigated the effect of a signal quantization technique on the performance of a binary BCI that uses a single, low-cost EEG electrode as input. We find that our technique allows for a BCI that is computationally efficient at training time, which can achieve good simulated accuracy for all subjects in our dataset, and boasts quick user calibration times. Specifically, we find that our quantization method decreases the computational expense of EEG-based calibration (from 18 ms to 2 ms for SVM training time) without a significant detriment to accuracy and, using quantized data, our progressive user calibration strategy achieves an average of 88.3\% accuracy across all subjects. All subjects required under 25 minutes of calibration time, and the system calibrated to all but one of these subjects in 15 minutes or fewer.

The conclusions to be drawn from this study are limited in a few regards. First, calibration and classification are performed offline, so factors involving the user interface (such as feedback) are not taken into account. We cannot be sure, for instance, that our findings based on the splicing of 10-second-long recorded data will persist when a system solicits recordings of only a second or under. Furthermore, a few of the gestures (e.g., the {\it color} labeled gesture) relied on exogenous stimuli, which may be impractical in naturalistic settings for ergonomic reasons. 

Our study indicates that practical BCI can be achieved with as few as one, inexpensive EEG sensor, minimal processing power, and a only a few minutes of user calibration. Future work could build usable, online BCI systems to test this claim more rigorously (e.g. on mobile computing platforms or in naturalistic settings). Since many types of bio-signals can be represented as time series of power spectra (e.g., electrocardiography, electromyography), future work could also test our quantization technique on different types of biometric signals.

Reducing the size of feature vectors in physiological computing applications could confer numerous benefits to application developers. Smaller feature vectors could enable quick, cloud-based processing, reducing the computational load on the end users' hardware. Small feature vectors also lower the boundary to achieving chronic, pervasive recording. By quantizing signals from physiological sensors, developers could collect large corpa of biometric data without expensive, high-performance server configurations, enabling observations on physiological data at large scale.


\section*{\uppercase{Acknowledgements}}
This research was supported in part by the National Science Foundation under award CCF-0424422 (TRUST) and the Swiss National Science Foundation under award PA00P2-145368

